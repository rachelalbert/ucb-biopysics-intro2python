
\documentclass[12pt]{article}


%%% PACKAGES

\usepackage{bibentry} %to use intext full bibliography entries instead of citations.  You will need a separate BibTex database for this to work.  See http://cst.usc.edu/services/tel/grants/legrants.html for details on this package.
\usepackage{booktabs} % for much better looking tables
\usepackage{array} % for better arrays (eg matrices) in maths
\usepackage{paralist} % very flexible & customisable lists (eg. enumerate/itemize, etc.)
%\usepackage{verbatim} % adds environment for commenting out blocks of text & for better verbatim
%\usepackage{subfigure} % make it possible to include more than one captioned figure/table in a single float


%%% PAGE DIMENSIONS
\usepackage{geometry} % to change the page dimensions. Read ftp://ftp.tex.ac.uk/tex-archive/macros/latex/contrib/geometry/geometry.pdf for detailed page layout information 
\geometry{margin=1in} % for example, change the margins to 1 inches all round
%\geometry{landscape} % set up the page for landscape
% 

%%% HEADERS & FOOTERS
\usepackage{fancyhdr} % This should be set AFTER setting up the page geometry
\pagestyle{fancy} % options: empty , plain , fancy
\renewcommand{\headrulewidth}{0.4pt} % customise the layout...
%\lhead{}\chead{}\rhead{}
%\lfoot{}\cfoot{\thepage}\rfoot{}

%\rfoot{\footnotesize SIR 330}
\rhead{\footnotesize SIR 330 Syllabus}
\renewcommand\footrulewidth{0pt}


%%% SECTION TITLE APPEARANCE
%\usepackage{sectsty}
%\allsectionsfont{\sffamily\mdseries\upshape} % (See the fntguide.pdf for font help)
% (This matches ConTeXt defaults)


%% END Article customise

%%% BEGIN DOCUMENT


\begin{document}


\thispagestyle{plain} %alternatively specify empty to get no footer on first page.  This is part of the fancyhdr package


\nobibliography{MasterBib} %this specifies the BibTex directory that stores your desired bibliography entries.  It has to come before any \bibentry lines are invoked

\bibliographystyle{apalike} %be careful here, there is only a few styles that will run


%\tableofcontents

\begin{center}
\LARGE{\bf{Introduction to Python}}

\textsc{UC Berkeley Biophysics, Fall 2014} \bigskip

\end{center}

\noindent\textbf{Instructors: }Rachel Albert \& Mike Schachter\medskip

\noindent\textbf{Time and Location:} T, Th 4:00-5:30pm, Location TBD\medskip

\noindent\textbf{Contact:} rachelalbert@berkeley.edu, mike.schacter@gmail.com\medskip

\bigskip

\section*{Overview and Objectives}%starred section will eliminate numbering; remove stars to get numbered sections especially if you are using TOC for some reason in your syllabus
This course is a one month introduction to the Python programming language. No prior programming experience is required. We will cover the following topics: how to set up your programming environment; how to use variables, data structures, conditional statements, loops, and functions; how to input and output data to and from Python; and an introduction to several useful scientific Python libraries.

\section*{Course Outline}

\subsection*{Installation and Setup \textnormal{\small{(9/2)} }}
\begin{itemize}
\item Course introduction
\item Installing Python
\item Setting up your environment
\item Package management \\\\
\it{Homework: Git tutorial}
\end{itemize}

\subsection*{Source control \textnormal{\small{(9/4)} }}
\begin{itemize}
\item Git overview
\item Accessing and turning in assignments
\item ??? \\\\
\it{Class project: *}\\
\it{Homework: Strings tutorial}
\end{itemize}

\subsection*{Input/Output \textnormal{\small{(9/9)} }}
\begin{itemize}
\item Command line interface
\item IPython \& IPython notebook
\item Variables (strings and numbers) \\\\
\it{Class project: Conditional statements}\\
\it{Homework: Data structures tutorial}
\end{itemize}

\subsection*{Manipulating data \textnormal{\small{(9/11)} }}
\begin{itemize}
\item Data structures overview
\item Loops \\\\
\it{Class project: }\\
\it{Homework: Functions tutorial}
\end{itemize}

\subsection*{Processing data \textnormal{\small{(9/11)} }}
\begin{itemize}
\item Functions
\item File I/O \\\\
\it{Homework: * tutorial}
\end{itemize}

\end{document} 